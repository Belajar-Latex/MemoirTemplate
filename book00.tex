% REV03 Mon Nov  5 18:04:10 WIB 2018
% START Tue May 22 19:45:45 WIB 2018

% (c) 2018 Rahmat M. Samik-Ibrahim 
% All Rights Reversed --- All Wrongs Ignored.
% This is a free document. 

\pagenumbering{roman}
\thispagestyle{empty}
\beforepartskip

\newlength{\centeroffset}
\setlength{\centeroffset}{-0.5\oddsidemargin}
\addtolength{\centeroffset}{0.5\evensidemargin}
%\addtolength{\textwidth}{-\centeroffset}
\thispagestyle{empty}
\vspace*{\stretch{1}}
\noindent\hspace*{\centeroffset}\makebox[0pt][l]{\begin{minipage}{\textwidth}
\flushright
{\Huge\bfseries \maintitle}
\noindent\rule[-1ex]{\textwidth}{5pt}\\[2.5ex]
\hfill\emph{\Large \subtitle}
\end{minipage}}

\vspace{\stretch{1}}
\noindent\hspace*{\centeroffset}\makebox[0pt][l]{\begin{minipage}{\textwidth}
\flushright
{\bfseries by 
\authora,\\[1.5ex]
\authorb,\\[1.5ex]
and 
\authorc\\[3ex]} 
Rev. \revision
\end{minipage}}

%\addtolength{\textwidth}{\centeroffset}
\vspace{\stretch{2}}

\pagebreak
\let\cleardoublepage\clearpage

\sloppy
\frontmatter
\begin{small} 
  \noindent Copyright \copyright 2018 \authora, \authorb, and \authorc.
  All rights reversed, all wrongs corrected.

  \lipsum[1]
\end{small}

\chapter{Thank you!}
\noindent%
\lipsum[1]

\newpage \noindent%
The following individuals helped with corrections, suggestions and
material to improve this book:

{\flushleft\small
\authora,
\authorb,
\authorc,
\authora,
\authorb,
\authorc,
\authora,
\authorb,
\authorc,
\authora,
\authorb,
\authorc,
\authora,
\authorb,
\authorc,
\authora,
\authorb,
\authorc,
\authora,
\authorb,
\authorc,
\authora,
\authorb,
\authorc,
\authora,
\authorb,
\authorc.
}

\vspace*{\stretch{1}}

\pagebreak

\chapter{Preface}
\lipsum[1-2]
\vspace{\stretch{1}}

\begin{verse} 
\begin{small} 
\noindent%
\authora{ }(\url{mailto:cbk@cbk.com}), \\
\noindent%
\authorb{ }(\url{https://cbk.vlsm.org/}), \\
\noindent%
and 
\authorc. \\
\noindent%
1600 Pennsylvania Ave NW, Washington, DC 20500,\\
\noindent%
America First.\\
\end{small} 
\end{verse} 
\clearpage

\tableofcontents
\clearpage

\listoffigures
\clearpage

\listoftables
\clearpage

\enlargethispage{\baselineskip}
\mainmatter
\pagenumbering{arabic}

%% % Original: bib.tex - A simple article illustrating the use of BibTex
%% % Andrew Roberts - June 2003
%% 
%% \documentclass{article}
%% \usepackage{natbib}
%% \bibpunct{(}{)}{,}{a}{,}{,}
%% \newcommand{\BibTeX}{{\sc Bib}\TeX}
%% 
%% \begin{document}
%% \author{Andrew Roberts}
%% \title{A Quick Look at \LaTeX}
%% \date{\today}
%% \maketitle
%% 
%% \section{Introduction}
%% \LaTeX{} is a typesetting system developed by Leslie
%% Lamport \citep{lamport94}.  It builds on foundations created by Donald
%% %% ZCZC 
%% Knuth's \TeX{} system \citep{knuth79}.  \TeX{} 
%% became very popular within
%% the scientiic community because it was very good at producing
%% mathematical manuscripts.  It was extremely powerful and provided the
%% user with exceptional control of the presentation of their documents.
%% In the 80s, Lamport began developing \LaTeX, which was designed to add a
%% layer of abstraction on top of \TeX{} which allows the user to focus
%% more on the document structure, rather than getting too bogged down with
%% presentation issues.  \LaTeX{} also added extra functionality through
%% auxiliary programs that can generate bibliographies, tables of contents,
%% indices, tables, cross-references and figures.
%% 
%% \section{How to learn more}
%% Here are some well established resources to help you learn more about
%% this excellent system.
%% 
%% \begin{itemize}
%% 	
%% 	\item General \LaTeX{} resources - the excellent \LaTeX{}
%%         %% ZCZC Companion \citep{goossens93} is a broad, yet in depth 
%%         look at the most important aspects.
%% 
%%         \item Graphics - take a look at Rahtz's 
%%         %% ZCZC	survey \citep{rahtz89} of graphics techniques in \TeX. 
%% 
%%         \item Bibliographies - the best place to start would be the
%%         %% ZCZC	\BibTeX{} documentation by \citet{patashnik88}.
%% 
%% 	\item Extras - The CTAN 
%%         %% ZCZC archives\citep{greenwade93} contain a vast
%%         number of supplimentary features, such as packages, macros, styles,
%%         etc., that can extend the potential of \LaTeX{} even further.
%% 
%% \end{itemize}
%% 
%% \bibliographystyle{plainnat}
%% \bibliography{rmsbib-list}
%% 
%% \end{document}
